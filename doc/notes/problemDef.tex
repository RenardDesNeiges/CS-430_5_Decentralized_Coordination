\documentclass[11pt]{article}

\usepackage{sectsty}
\usepackage{graphicx}
\usepackage[T1]{fontenc}
\usepackage{epigraph} %quotes
\usepackage{amssymb} %math symbols
\usepackage{mathtools} %more math stuff
\usepackage{amsthm} %theorems, proofs and lemmas
\usepackage[ruled,vlined,linesnumbered]{algorithm2e} %algoritms/pseudocode

%% Theorem notation
\newtheorem{theorem}{Theorem}[section]
\newtheorem{corollary}{Corollary}[theorem]
\newtheorem{lemma}[theorem]{Lemma}
\newtheorem{problem}{Problem}[section]

%% declaring abs so that it works nicely
\DeclarePairedDelimiter\abs{\lvert}{\rvert}%
\DeclarePairedDelimiter\norm{\lVert}{\rVert}%

% Swap the definition of \abs* and \norm*, so that \abs
% and \norm resizes the size of the brackets, and the 
% starred version does not.
\makeatletter
\let\oldabs\abs
\def\abs{\@ifstar{\oldabs}{\oldabs*}}
%
\let\oldnorm\norm
\def\norm{\@ifstar{\oldnorm}{\oldnorm*}}
\makeatother

% Marges
\topmargin=-0.45in
\evensidemargin=0in
\oddsidemargin=0in
\textwidth=5.5in
\textheight=9.0in
\headsep=0.5in


\title{Biding Algorithm - Problem Definition}
\date{\today}
%\author{Name}

\begin{document}
\maketitle	

\subsection{General problem description}

Our goal is to maximize our profit margin $M_p$ over all $n$ elements in the set $T$ of all $t_i$ auctions. Where $r(t_i)$ is the revenue from an auction ($r(t_i)=0$ of if the auction is lost) and $c(t_i)$ is the cost that carrying out an auctioned contract ($r(t_i)=0$ of if the auction is lost). In an adversarial setting where the contract is awarded to the lowest bidding agent $a$ (\textit{closed-bid first-price reverse auction}). Where $a\in A$ the set of all agents in the game. Our own agent is denoted as $\alpha \in A$.

\begin{align}
    M_p = \sum_{t_i \in T} r(t_i) - \sum_{t_i \in T} c(t_i)
\end{align}

At each step $i$ our profit margin varies by a values $\Delta M_p(i)$ :

\begin{align}
    \Delta M_p(i) = r(t_i) - c(t_i)
\end{align}

For a given \textit{task} our decision algorithm has the following informations : 
\begin{align*}
    \text{Cost of adding a given task to the plan : } && c(t_i) && \forall t_i \\
    \text{Price offered by every agent for previous bids : } && p(t_i,a) && \forall t_i,a\\
    \text{Cost of carrying out auction for previous bids : } && c(i) && \forall i \\
    \text{Winner of a given bid : } && w(i) && \forall i
\end{align*}

The problem can essentially be thought of as a \textit{multi-armed bandit} problem with an exploration/exploitation tradeoff, we will use a variation on \textit{Thomson Sampling}. The general idea is to fit a probability distribution to estimate our winning probability at each bid. And then pick a bidding amount from that probability distribution in such a way that our expected winning value $\mathbb{E}\left(\Delta M_p(i)\right)$ is maximal (greedy algorithm). Or to pick a value at random out of a distribution constructed in such a way that our expected winning value is maximized over time but we don't give up on exploration, which actually tends to yield better results than the greedy solution (\textit{Thomson Sampling}).

\section{Defining error}

We define the error made for a given auction as follows :
\begin{align}
    e(i) = \min_{\forall a \in A/ \alpha} \left\lbrace p(t_i,a) \right\rbrace - p(t_i,\alpha) 
\end{align}

For auctions that our agent won we have $e(i_{won})\geq 0$, for auctions lost we have $e(i_{won})\leq 0$. 




\begin{algorithm}[H]
    \SetAlgoLined
    \caption{Thompson Sampling For Automatic Bidding}
    \Repeat(){no task remains to be auctioned}{
        Compute 
    }
\end{algorithm}

\end{document}