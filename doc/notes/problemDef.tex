\documentclass[11pt]{article}

\usepackage{sectsty}
\usepackage{graphicx}
\usepackage[T1]{fontenc}
\usepackage{epigraph} %quotes
\usepackage{amssymb} %math symbols
\usepackage{mathtools} %more math stuff
\usepackage{amsthm} %theorems, proofs and lemmas
\usepackage[ruled,vlined]{algorithm2e} %algoritms/pseudocode

%% Theorem notation
\newtheorem{theorem}{Theorem}[section]
\newtheorem{corollary}{Corollary}[theorem]
\newtheorem{lemma}[theorem]{Lemma}
\newtheorem{problem}{Problem}[section]

%% declaring abs so that it works nicely
\DeclarePairedDelimiter\abs{\lvert}{\rvert}%
\DeclarePairedDelimiter\norm{\lVert}{\rVert}%

% Swap the definition of \abs* and \norm*, so that \abs
% and \norm resizes the size of the brackets, and the 
% starred version does not.
\makeatletter
\let\oldabs\abs
\def\abs{\@ifstar{\oldabs}{\oldabs*}}
%
\let\oldnorm\norm
\def\norm{\@ifstar{\oldnorm}{\oldnorm*}}
\makeatother

% Marges
\topmargin=-0.45in
\evensidemargin=0in
\oddsidemargin=0in
\textwidth=5.5in
\textheight=9.0in
\headsep=0.5in


\title{Biding Algorithm (problem definition)}
\date{\today}
%\author{Name}

\begin{document}
\maketitle	

Our goal is to maximize our profit margin $M_p$ over all $n$ elements in the set $T$ of all $t_i$ auctions. Where $r(t_i)$ is the revenue from an auction and $c(t_i)$ is the cost that carrying out an auctioned contract. In an adversarial setting where the contract is awarded to the lowest bidder (\textit{closed-bid first-price reverse auction}).

\begin{align}
    M_p = \sum_{t_i \in T} r(t_i) - \sum_{t_i \in T} c(t_i)
\end{align}

For a given \textit{task} our agent computes the following informations : 
\begin{align*}
    \text{Cost of adding a given task to the plan : } && c(t_i) \\
\end{align*}

\end{document}